\begin{enumerate}
\item Structure from Motion

   VisualSfM:
   \begin{verbatim}
     1. Load images:               File -> Open+ Multiple Images
     2. Calc+Match SIFT features:  SfM  -> Pairwise Matching -> Compute Missing Match
     3. Reconstruct (Bundle Adj):  SfM  -> Reconstruct 3D
     4. Save sparse point cloud:   SfM  -> Save NV Match (.nvm)
     5. Dense (optional):          SfM  -> Run CMVS/PMVS
   \end{verbatim}

   Voodoo:
   \begin{verbatim}
     1. Load images:               File -> Open -> Sequence
     2. Calc+Track features:       Track (bottom panel)
     3. Save:                      File -> Save -> Textfile (.txt)
   \end{verbatim}

   Bundler:
   \begin{verbatim}
     1. Run provided script from within direct of images (Cygwin)
     2. Result is saved in bundle/bundle.out
   \end{verbatim}


\item Visualisation of SfM results

   \begin{verbatim}
   ./sfmviewer <sfm> [<images>]
   where sfm is the path to an nvm, ply, out or txt file,
   and images is the path to the corresponding directory with images.
   \end{verbatim}

   Select points or cameras with Shift + Click. \\
   Enable/disable line drawing with 'd'. \\
   More shortcuts listed with 'h' (PCL) and '?' (custom).


\pagebreak
\item Space Carving

   \begin{verbatim}
   ./sfmcarver <sfm> <imgs> <save> [<method> [<resolution> [<param1>]]
     sfm:    path to an nvm, ply, out or txt file
     imgs:   path to the directory containing the images
     save:   filename for saving the octree (.ot, or .bt for binary)
     method: carve method (0-3); default: 2;
             0=discretised, 1=vis, 2=vis+occ veto, 3=vis+occ+extend vis lists
     resol.: voxelgrid sizes (determines smallest octree node size); default: 250
     param1: first parameter of given method; default: 0.1
   \end{verbatim}


\item Visualisation of Space Carve results

   \begin{verbatim}
   octovis <carve>
     where carve is an octree file (.ot or .bt)
   \end{verbatim}

   or:

   \begin{verbatim}
   ./carveviewer <sfm> <imgs> <carve>
     where sfm is the path to an nvm, ply, out or txt file
     imgs is the path to the directory containing the images
     and carve is an octree file (.ot or .bt)
   \end{verbatim}
   

\item Optional: regularisation

   \begin{verbatim}
   ./octreegraphcut <octree in> <octree out> [<gamma> [<unknown>]]
     octree in/out: filename of octree file (.ot, or .bt for binary)
     gamma:         weight of voxel prob difference for pairwise cost; default: 1
     unknown:       occupancy probability of unknown (un-initialised) voxels;
                    default: 0.2
   \end{verbatim}


\item Optional: feature visualisation

   \begin{verbatim}
   ./featureviewer <path> [<pointDetector> [<pointDescriptor> [<edgeDetector>]]]
     where path is the path to an AVI file, directory containing
     a sequence of images, or webcam device number;
     pointDetector: {SIFT, SURF, ORB, FAST, STAR, MSER, GFTT, HARRIS,
                     Dense, SimpleBlob};
     pointDescriptor: {SIFT, SURF, ORB, BRIEF};
     edgeDetector: {CANNY, HARRIS}
   \end{verbatim}

\end{enumerate}
